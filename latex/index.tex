간략한 설명 자세한 설명 \hypertarget{index_intro}{}\section{Introduction}\label{index_intro}
Tizen 2.\-3의 Native Application A\-P\-I는 필요한 정보들이 세분화 분리가 되어있다. 예를 들어 N\-F\-C를 이용하여서 기본적인 데이터를 작성하는 Application을 개발 해야 한다고 가정 하면, N\-F\-C를 기록하기 위해서는 기본적인 N\-F\-C 장비의 유무를 체크하고(\-Sensor), N\-F\-C 장비가 활성화 상태인지 확인하고(\-System Information), N\-D\-E\-F 형태의 데이터 기록을 준비하고(\-N\-D\-E\-F), N\-F\-C를 기록(\-N\-F\-C) 하는 형태로 진행이 된다. 각각의 A\-P\-I가 독립적으로 있기 때문에 개발자는 각 각의 A\-P\-I를 통해서 접근 해야 한다. Tizen Application을 개발할 때 한 기능을 구현 하기 위해서는 여러 A\-P\-I를 융합하면서 사용한다. 따라서 Tizen Application을 개발을 할 때는 기능을 사용하기 위한 A\-P\-I와 기능을 수행하는 데 필요한 역할을 하는 A\-P\-I를 알아야 한다는 부담이 존재한다. D\-I\-T의 A\-P\-I는 다음과 같이 3개로 구분된 구조를 제공 함으로써 기존에 나눠서 제공 되어있는 A\-P\-I를 묶어서 제공한다. 또한, Tizen A\-P\-I를 사용하면서 직관적이지 못한 함수명과 함수의 매개변수 전달 시의 복잡한 문제를 해결을 하여 사용하기 쉬운 구조로 변경을 한다. \hypertarget{index_Program}{}\section{프로그램명??}\label{index_Program}

\begin{DoxyItemize}
\item 프로그램명 \-: D\-I\-T라고 허허...여기 앞을 어떻게 채울지 핵고민중
\item 프로그램내용 \-: Tizen Native A\-P\-I easy version 
\end{DoxyItemize}\hypertarget{index_CREATEINFO}{}\section{작성정보}\label{index_CREATEINFO}

\begin{DoxyItemize}
\item 작성자 \-: 최똥
\item 작성일 \-: 2015-\/07-\/24 \begin{DoxyDate}{Date}
2015-\/07-\/24 
\end{DoxyDate}

\end{DoxyItemize}\hypertarget{index_MODIFYINFO}{}\section{수정정보}\label{index_MODIFYINFO}

\begin{DoxyItemize}
\item 수정일 / 수정자 \-: R\-E\-A\-D\-M\-E.\-md처럼 꾸며야 할듯
\item 멘붕... 

 
\end{DoxyItemize}